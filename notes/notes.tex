\documentclass{ctexart}
\usepackage{CJK}
\usepackage{graphicx}
\usepackage{bm}
\usepackage{subfigure}
\usepackage{multirow}
\usepackage{algorithm}
\usepackage{algorithmic}
\usepackage{array}
\usepackage{subfigure}
\usepackage{mathrsfs}
\usepackage{ulem}
\usepackage{float}
\usepackage{epstopdf}
\usepackage{subfigure}
\usepackage{caption}
\usepackage{amssymb}
\usepackage{amsmath}
\begin{document}
\setlength{\parindent}{0pt}
\begin{center}
  \uline{\textbf{实验笔记}}
\end{center}
\section{数据概要}
\subsection{老鼠数据}
\begin{table}[h]
\caption{Notations}
\label{sample-table}
\begin{center}
\begin{tabular}{lll}
\multicolumn{1}{c}{\bf PART}&\multicolumn{1}{c}{\bf VALUE}  &\multicolumn{1}{c}{\bf DESCRIPTION}
\\ \hline \\
$n_1$    &1376      &基因-表型 关联矩阵中的基因数目 \\
$m$    &5420      &基因-表型 关联矩阵中的表型数目 \\
$n_2$    &5985      &PPI 基因-基因 关联矩阵中的基因数目 \\
$n_3$    &7754      &pathway中所有出现的基因数目 \\
$n_1\cap n_2$    &670      &基因-表型 关联矩阵与PPI网络中共有的基因数目 \\
$n_1\cap n_3$    &790      &基因-表型 关联矩阵与pathway中共有的基因数目 \\
$n_1\cap n_2\cap n_3$    &441      &基因-表型 关联矩阵,PPI,pathway,中共有的基因数目 \\
\\ \hline \\
\end{tabular}
\end{center}
\end{table}

获取基因的原则:基因或者出现在PPI网络中,或者出现在pathway中,这样在目标函数中便有约束可以作用在所出现的基因上面。\\

\begin{itemize}
\item 如果选择670个基因作为基准基因集合,则会有229(670-441)个基因没有出现在任何pathway中,在计算目标函数与梯度过程中,这229个基因并不影响计算,在遍历pathway时,只考虑那441个基因便可。因此我们应将每个pathway下的基因重新筛选一次,每条pathway下,只用这441个基因来表示即可。
\item 如果选择790个基因作为基准基因集合,则会有349(790-441)个基因没有出现在PPI网络中,此时PPI网络要保留为$790\times 790$大小,也就是说在PPI网络中有349个基因与其他基因是没有任何关联的。
\end{itemize}



%\begin{CJK}%{GBK}{song}
%收到
%\end{CJK}
\section{具体实验流程}
选择 670 个基因作为基准基因集合的实验流程:
\begin{algorithm}[h]
\caption{\textbf{具体实验流程}}\label{alg:CMNMF}
\renewcommand{\algorithmicrequire}{\textbf{Input:}}
\renewcommand{\algorithmicensure}{\textbf{Output:}}
\label{alg:pf}
\begin{algorithmic}[1]
\STATE 获取670个基因的基因列表,用这1到670作为基因索引
\STATE 用新的670个基因索引表示 基因-表型网络和基因PPI网络
\STATE 在670个基因索引中进一步挑选出441个基因的索引列表,用于表示每个pathway下的基因
%\REQUIRE
%    \STATE $G$: hierarchical HIN;
%    \STATE $\mathcal{P}$: the set of meta paths connecting genes;
%    \STATE $\lambda_1$,$\lambda_2$: adapting parameter;
%    \STATE $\alpha$: step size;
%    \STATE $\epsilon$: convergence tolerance;
%\ENSURE ${\bm{W}}$: the weight matrix of all genes on all paths
%\STATE $P \leftarrow \{(i,j)|q_i=q_j,y_i\succ y_j\}$
%\FOR{{each meta path $\mathcal{P}_l$ $\in$ $\mathcal{P}$}}
%    \STATE {Evaluate user similarity $\bm{S}^l$}
%    \STATE {Calculate predicting score $\hat{\bm{R}^l}$ with Eq.()}
%    %\STATE \texttt{}
%\ENDFOR
%\STATE ${\bm{W}}$ $\leftarrow$ random values(between [0,1])
%%\STATE \text{stop$\_$condition:iterate number is more than $N$ or the change of objective function's value is less than $\varepsilon$}
%%\STATE $ite = 1$
%\REPEAT
%    \STATE Calculate $\frac{\partial{\mathcal{L}}}{\partial{(\bm{W})_{ij}}}$ with Eq.()
%    \STATE Update $\bm{W}_{ij}\leftarrow max(0,\bm{W}_{ij}-\alpha \frac{\partial{\mathcal{L}}}{\partial{(\bm{W})_{ij}}})$
%    %\STATE Normalize $(\bm{P}_1)_{ij}\leftarrow \frac{(\bm{P}_1)_{ij}}{\sum_{i}(\bm{P}_1)_{ij}}, \quad
%%(\bm{P}_2)_{ij}\leftarrow \frac{(\bm{P}_2)_{ij}}{\sum_{i}(\bm{P}_2)_{ij}}$
%\UNTIL convergence
%\RETURN ${\bm{W}}$
\end{algorithmic}
\end{algorithm}

选择 790 个基因作为基准基因集合的实验流程:
\begin{algorithm}[h]
\caption{\textbf{具体实验流程}}\label{alg:CMNMF}
\renewcommand{\algorithmicrequire}{\textbf{Input:}}
\renewcommand{\algorithmicensure}{\textbf{Output:}}
\label{alg:pf}
\begin{algorithmic}[1]
\STATE 获取790个基因的基因列表,用这1到790作为基因索引
\STATE 用新的790个基因索引表示 基因-表型网络和基因pathway
\STATE 在790个基因索引中进一步挑选出441个基因的索引列表,用于表示PPI网络,实际上在790个基因组成的PPI网络中
,有349个基因与其他基因是没有任何关联的
\end{algorithmic}
\end{algorithm}
\end{document}
